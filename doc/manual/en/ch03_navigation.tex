\chapter{Navigation}\label{cha:navigation}
This chapter describes the moving map display as an aid to navigation,
and also describes some of the task and glide related overlays on the
map display.

\section{Map display elements}

\begin{maxipage}
\includegraphics[angle=0,width=0.9\linewidth,keepaspectratio='true']{figures/fig-map.png}
\end{maxipage}

The moving map shows:
\begin{enumerate} 
\item Glider, wind indicator, thermal profile, final glide indicator
\item Terrain, relief and hight of the terrain
\item Topography, rivers, roads, towns
\item Waypoints, airports, landables
\item The active task, observation zones, turnpoints
\item The bearing (or route\footnote{The line to the next waypoint may be a 
  \emph{route}, as described in Section~\ref{sec:route}.}) to the next waypoint, 
  heading
\item Airspaces
\item Markers, thermals history, snail trail
\item Glide range\footnote{The glide range is also referred to as the 
  \emph{reach}, as described in Section~\ref{sec:reach}.}
\end{enumerate}
The map is drawn in a projected coordinate system (not latitude and
longitude), and the scale can be changed (zooming in and out), as well
as panned.  All navigation functions take the curvature of the Earth
into account.

\section{Glider symbol, map orientation}
The glider symbol shows the position of the glider on the map.  The
orientation of the glider indicates the estimated heading of the
glider.

The map is oriented in one of three ways: North up,
Track up, or Target up.  Configuration settings \config{orientation} can be used
to specify a different map orientation when in circling mode. This is useful to prevent
disorientation when looking at the map while circling.  Target-up when
circling makes it easy to determine which direction to exit the
thermal.

When Track or Target-up is used in circling mode, the glider symbol is
centred on the screen, even if the symbol position is configured differently.
In cruise mode the Track and the Target-up orientation allows the glider
symbol to be positioned (e.g.) 20\% from the bottom of the screen, giving a good view of the
map ahead of the glider.  This position is adjustable in the configuration
\config{gliderposition} settings.

\section{Zoom and map scale}\label{sec:zooming}

To change the scale of the map, depending on the hardware you use:
\begin{enumerate}
\item Tap/click on a blank part of the map to highlight the map if it is not
already selected.
Then use mouse wheel, or the Pocket PC up/down key to either zoom
in or out.
\item Android devices have the +/- rocker that let you change the zoom (it usually allows for adjusting the sound volume). It may be difficult to access during a flight, when the device in on a generic mount.
\item You can also gesture to change the zoom level. Gesture 
Up (see left) zooms in, Down zooms out.\gesturespec{up}
\item Or select the function from the menu.
\begin{quote}
\bmenug{Display 1}\blink\bmenug{Zoom In} and \bmenug{Zoom Out}
\end{quote}
\end{enumerate}

The map scale is displayed in the lower left corner of the moving map
display. The indicated distance is measured from the left to the right border
of the map display.
\marginpar{\includegraphics[angle=0,width=0.4\linewidth,keepaspectratio='true']{figures/zoom.png}}

Compaq Aero Users. If you enable the Compaq Aero Game Keys (On the
Q-menu) the centre two front buttons become the up/down keys.

\subsection*{Circling Zoom}
There is a facility to have two zoom settings; one when the glider is
in circling mode, and one in the cruise or final mode.  This is the `\emph{Circling zoom}' 
option in the \config{circlingzoom} configuration settings.  
By default, the circling zoom is set to about 2.5 km - 5.0 km, depending on the
display size. When the user zooms in or out, it affects the current
mode's zoom setting only, so when leaving the mode the previous mode's
zoom setting is used.  If `\emph{Circling Zoom}' is not enabled,
there is only a single zoom level.
This leads to different zoom levels being preserved to be managed manually, 
independent of the following Auto Zoom settings.

\subsection*{Auto Zoom}
Auto Zoom automatically zooms in when approaching a waypoint to keep
the waypoint at a reasonable screen distance. When \emph{Auto Zoom} is active,
`AUTO' appears next to the map scale.
\marginpar{\includegraphics[angle=0,width=0.4\linewidth,keepaspectratio='true']{figures/zoomauto.png}}

To invoke Auto Zoom use the gesture \gesturespec{ud}
or menu path depicted to the left. 
Switching back to manual zoom is simply done by using the same menu path
or just zooming manually, no matter whether done via menu or gesture.
\menulabel{\bmenug{Display 1}\blink\bmenut{Zoom}{Auto}}

When a waypoint changes (automatically, via the task selector, or by
manually switching waypoints), \emph{Auto Zoom} adjusts the zoom level
automatically so that the next waypoint is visible on the map.

During circling, if a thermal has been detected, then the map is centred about
the thermal or part-way such that the glider is still visible.

\section{Panning the map}\label{sec:panning}

A pan mode allows the user to explore areas beyond the glider.  This
is particularly useful when task planning.
\begin{enumerate}
\menulabel{\bmenug{Display 1}\blink\bmenug{Pan On}}
\item Enable pan mode by button menu or by gesture.  The gesture for this is moving your fingertip up, right, down, left, supposed to form a ``P''.
\gesturespec{urdl}
\item The map can then be panned by dragging the screen or using the cursor
  keys.
\item When done, pan mode has to be disabled manually, by pressing `\emph{Pan Off}' 
  from the special sub-menu of buttons in pan mode.
\end{enumerate} 

\sketch{figures/pan.png}
When pan is active, the letters `PAN' appears next to the map scale.  While
panning the location of the focus stays in the middle of the display under the
cross hairs. 

Despite the focus under the cross-hairs the map
still offers the `\emph{What's here?}' feature just by touching any 
position on the map (presuming a touch-screen).


\section{Waypoints}\label{sec:waypoint-schemes}
Waypoints are displayed with different symbols depending on the
waypoint type; the major distinction being landable and non-landable
waypoints.

\subsection*{Landables}
The waypoint symbols are drawn as shown below. There are three icon sets for
landable waypoints. \config{waypointicons}

\begin{tabular}{c|ccc|ccc|}
Icon set 
&\begin{sideways}Landable field\end{sideways}
&\begin{sideways}Marginal\end{sideways}
&\begin{sideways}Reachable\end{sideways}
&\begin{sideways}Airfield\end{sideways}
&\begin{sideways}Marginal\end{sideways}
&\begin{sideways}Reachable\end{sideways}\\
\hline
Purple Circle &
\includegraphics[width=0.8cm]{icons/winpilot_landable.pdf} &
\includegraphics[width=0.8cm]{icons/winpilot_marginal.pdf} &
\includegraphics[width=0.8cm]{icons/winpilot_reachable.pdf} &
\colorbox{white}{\includegraphics[width=0.8cm]{icons/winpilot_landable.pdf}} &
\includegraphics[width=0.8cm]{icons/winpilot_marginal.pdf} &
\includegraphics[width=0.8cm]{icons/winpilot_reachable.pdf} \\
\hline
B/W & 
\includegraphics[width=0.9cm]{icons/alt_landable_field.pdf} &
\includegraphics[width=0.9cm]{icons/alt_marginal_field.pdf} &
\includegraphics[width=0.9cm]{icons/alt_reachable_field.pdf} &
\colorbox[rgb]{0.94,0.94,0.94}{\includegraphics[width=0.9cm]{icons/alt_landable_airport.pdf}} &
\includegraphics[width=0.9cm]{icons/alt_marginal_airport.pdf} &
\includegraphics[width=0.9cm]{icons/alt_reachable_airport.pdf} \\
\hline
Traffic lights & 
\includegraphics[width=0.9cm]{icons/alt2_landable_field.pdf} &
\includegraphics[width=0.9cm]{icons/alt2_marginal_field.pdf} &
\includegraphics[width=0.9cm]{icons/alt_reachable_field.pdf} &
\colorbox{white}{\includegraphics[width=0.9cm]{icons/alt2_landable_airport.pdf}} &
\includegraphics[width=0.9cm]{icons/alt2_marginal_airport.pdf} &
\includegraphics[width=0.9cm]{icons/alt_reachable_airport.pdf} \\
\hline
\end{tabular}

The \emph{marginal} icons are drawn for those waypoints which are principally in the 
reach, but it is not possible to approach them directly. E.g.\ a mountain prohibits a direct approach.
  
Waypoints are optionally labelled according to one of several
abbreviation schemes \config{labels} and visibility.

On top of this landable waypoints can be displayed in more detail. If
`\emph{Detailed landables}' is switched on you get additional information
encoded in the appearance of it's icon. 
\begin{enumerate}
\item  Landable fields get a square-shaped icon despite what is shown in the table.
  The square is drawn like a diamond standing on one corner. Airfields stay with the 
  circle shape, so that they become easy to distinguish.
\item  All icon sets, including the `Purple Circle' icon set, get the 
  runway turned into their actual direction. The runway direction has to be available in 
  the waypoint data. E.g.\ the SeeYou waypoint file format (\verb|.cup|) does
  include this information.  
\end{enumerate}

\subsection*{Landables in Reach}
Next to landables an estimated arrival height
\emph{above the arrival safety height of reachable landable} points is
displayed next to the waypoint. This feature is one of the most powerful out 
of XCSoar's capabilities. The arrival height is calculated highly 
configurable by XCSoar's \emph{glide computer} with parameters taken into 
account being glider performance (polar), MacCready setting, wind, terrain 
clearance, and --- obviously --- safety heights' values.
With all of it being 
configurable, there is enough room for failure, so please:
Unless you will have fully understood the glide computer's concepts, you 
\warning
better stay with XCSoar's pre-configuration (and in no way judge readings as 
heavenly approved).
It is up to the pilot to always interpret readings and watch values trending 
over time.

However, having set up the glide computer following
Chapter~\ref{cha:glide} the \emph{display} of estimated reach heights, drawn beside 
landables in reach may take into account terrain or not or display both.
\config{arrivalheight}

Another option is to display the required glide ratio next to a landable in 
reach. This calculation is simply derived by the glider's actual distance to 
landables divided by the height difference between actual altitude and 
landable's altitude.  Again, the safety height is added to the landable's 
height, but nothing else taken into account: no wind, no polar, no MacCready 
settings, just geometry. The concept of required glide ratios is a widely 
discussed concept, said as to be a very robust one.

\tip Keep in mind a strong relationship of displays of \emph{reach} and 
settings in the glide computer.

\subsection*{Non-Landables}
As far as your waypoint file contains information on the nature of the 
non-landable waypoints, the map will then display specific icons accordingly. 
Figure~\ref{fig:nonlandables} contains a collection of the currently supported map icons.

\begin{figure}[htbp]
\centering
\vspace{2.5cm}
\begin{tabular}{ccccccccc}
\begin{rotate}{60}Simple waypoint\end{rotate} &
\begin{rotate}{60}Mountain top\end{rotate} &
\begin{rotate}{60}Obstacle\end{rotate} &
\begin{rotate}{60}Pass\end{rotate} &
\begin{rotate}{60}Power plant\end{rotate} &
\begin{rotate}{60}Tower or building\end{rotate} &
\begin{rotate}{60}Tunnel\end{rotate} &
\begin{rotate}{60}Weather station\end{rotate} &
\begin{rotate}{60}Bridge\end{rotate}\\

\includegraphics[width=0.5cm]{icons/map_turnpoint.pdf} &
\includegraphics[width=0.8cm]{icons/map_mountain_top.pdf} &
\includegraphics[width=0.7cm]{icons/map_obstacle.pdf} &
\includegraphics[width=0.7cm]{icons/map_pass.pdf} &
\includegraphics[width=0.8cm]{icons/map_power_plant.pdf} &
\includegraphics[width=0.7cm]{icons/map_tower.pdf} &
\includegraphics[width=0.6cm]{icons/map_tunnel.pdf} &
\includegraphics[width=0.6cm]{icons/map_weather_station.pdf} &
\includegraphics[width=0.8cm]{icons/map_bridge.pdf}
\end{tabular}
\caption{non-landables}\label{fig:nonlandables}
\end{figure}

\section{Active task}

The active task course is drawn on the map as a blue (dashed) line.

Assigned area tasks also show the turn point sectors or areas as a yellow shaded
region.  Circles are always drawn around start and finish points, lines are
only drawn if the start/finish points are of line type.  
Task observation sectors are drawn as circle segments.

At all times a thick black line is drawn from the glider to the next
waypoint in the task.  This line may be the direct path to the waypoint,
or may be a \emph{route} path clearing terrain and airspace obstacles, described in
further detail in Section~\ref{sec:route}.

\begin{center}
\begin{tabular}{c c c}
\emph{Start/finish} & \emph{Sector} & \emph{Cylinder} \\
\includegraphics[angle=0,width=0.3\linewidth,keepaspectratio='true']{figures/cut-startfinish.png} &
\includegraphics[angle=0,width=0.3\linewidth,keepaspectratio='true']{figures/cut-sector.png} &
\includegraphics[angle=0,width=0.3\linewidth,keepaspectratio='true']{figures/cut-barrel.png}
\end{tabular}
\end{center}


\section{Terrain and Topography}\label{sec:terrain_topo}

The following topographical features are drawn on the map:
\begin{itemize}
\item Major roads, shown as red lines
\item Rivers, shown as blue lines
\item Large water bodies (lakes), shown as blue areas
\item Large cities, shown as yellow areas
\item Small population areas, shown as yellow diamonds
\end{itemize}
Cities and small population areas are labeled in italics.

Terrain is coloured according to elevation, and optionally shaded by sun, or 
wind direction.  Invalid terrain, or terrain below
sea level is coloured blue.

\menulabel{\bmenug{Display 2}\blink\bmenut{Terrain}{On/Off}}
\menulabel{\vspace{1cm}\bmenug{Display 2}\blink\bmenut{Topo.}{On/Off}}

Terrain is shaded to improve visibility.  The default shading
is set up so that the virtual lighting position is the wind bearing,
thus brighter areas are on the upwind side of hills and dark areas in
the lee of the hill.  
Support for a sun ephemeris is also implemented. If the slope shading is set 
to `Sun', the brightness of a slope follows the day time in a very natural way.
The amount of shading and overall terrain brightness is configurable. \config{shading}

Both terrain and topography display can be switched on or off from the
menu.

\begin{center}
\begin{tabular}{c c}
Topography & Terrain \\
\includegraphics[angle=0,width=0.4\linewidth,keepaspectratio='true']{figures/cut-topo.png} &
\includegraphics[angle=0,width=0.4\linewidth,keepaspectratio='true']{figures/cut-terrain.png}
\end{tabular}
\end{center}

If the terrain data is not available (or terrain display is turned
off), the background colour of the map window is white.  All terrain
below mean sea level is coloured blue.  If you are flying outside the
terrain region, the background colour will be white.

\subsection*{Map labels}\label{sec:maplabels}

The screen can be de-cluttered, turning off the display of all waypoint labels by toggling the `\emph{Labels}' menu.
\menulabel{\bmenug{Display 2}\blink\bmenut{Labels}{None}}

Other options for display decluttering include:

\jindent{\bmenuth{Labels}{Task \&}{Airfields}}{ Shows labels for the waypoints in 
  the active task and any airfields (based on the waypoint attributes in 
  the waypoints file).  Other waypoints are shown but not labeled. }
\jindent{\bmenut{Labels}{All}}{ Shows labels for all waypoints. }
\jindent{\bmenuth{Labels}{Task \&}{Landables}}{  Shows labels for the waypoints in 
  the active task and any landable fields (based on the waypoint attributes in 
  the waypoints file).  Other waypoints are shown but not labeled. }
\jindent{\bmenut{Labels}{Task}}{ Shows labels only for waypoints in the active task}

Note that in all cases, the label format is configurable in the 
`\emph{Waypoint Display}' configuration menu.  \config{labels}


\section{Trail}\label{sec:trail}

An optional `snail trail' is drawn on the map showing the glider's
path history.  The colour and thickness of the trail depends on the altitude or
on the variometer value. 

\begin{center}
\includegraphics[angle=0,width=0.5\linewidth,keepaspectratio='true']{figures/snail.pdf}
\end{center}

If Vega or an intelligent variometer is connected with Netto output,
the Netto vario value is used; hence the colours and thickness of the
trail indicates the air-mass vertical movement rather than the glider's
vertical movement.

\config{snailtrail}
The snail trail display can be toggled between \emph{Off}, a \emph{Short} trail
(about ten minutes), a \emph{Long} trail (about one hour) or a \emph{Full} trail
which displays the entire flight.  This can be performed permanently
through the configuration settings or temporarily by the
menu.
\menulabel{\bmenug{Display 2}\blink\bmenut{Trail}{Full}}

Note that for all of these modes, the snail trail is short in
circling mode in order to reduce screen clutter.

In order to assist centring thermals in the presence of wind, the
snail trail can be artificially drifted with the wind as it is
displayed (this is drift compensation).  In this way, the snail trail
is referenced to the prevailing wind rather than referenced to the
ground.  Since thermals drift with the wind also, the drifted trails
give a better indication of where the glider has been relative to the
thermals.

An example of this is illustrated below.  Note that when trail drift
compensation is active (right picture), the glider appears to be
circling in a column rather than an elongated spiral (left picture).

\begin{center}
\includegraphics[angle=0,width=0.6\linewidth,keepaspectratio='true']{figures/traildrift.png}
\end{center}

\config{traildrift}
Enabling trail drift compensation is performed through the
configuration settings.  The compensation is only performed
whilst in circling mode; the display of the trail in cruise mode is unaffected.
This can also be performed from the wind settings dialogue.
\menulabel{\bmenug{Config 1}\blink\bmenus{Wind}}

The trail drift display is useful also to show more clearly when thermals
are cranked due to wind shear.

The trail width can optionally be scaled according to the variometer display\config{trailscaled}.


\section{Markers}\label{sec:markers}

Markers are shown as small flags (a) on the map.  The markers can be dropped
manually, by pressing a button, or automatically.  An example use of
automatic markers is to drop markers when entering circling mode, as a
simple way of showing all thermals encountered.

\menulabel{\bmenug{Nav 2}\blink\bmenut{Mark}{Drop}}
Markers are not preserved after XCSoar is exited, however the location
of all marks are appended to the file \verb|xcsoar-marks.txt|.

\section{Pilot Event}\label{sec:pilotevent}

A Pilot Event (PEV) system allows pilots to mark particular times during a
flight, such when inside a Turn Point Observation Zone (TP OZ). Also, when
configured, it allows to initiate a special \emph{Pilot Event} task start
protocol\footnote{PEV start procedure may be used in competitions to reduce
mass starts, gaggles and following.}.

\menulabel{\bmenug{Nav 2}\blink\bmenut{Pilot}{Event}}

When pressed, the task start window will be automatically set according to
configuration defined in Task Rules (See \ref{sec:task-type-racing}): start
window will open after \emph{PEV start wait time} \config{taskrules} minutes
and will be open for \emph{PEV start window} minutes.  Also, \emph
{Pilot Event} will be announced to connected devices, allowing loggers to
register it without pilot pressing the physical button on the logger device
itself. This might be useful when logger is located outside of pilot reach
(e.g. behind the panel).

In order to start a task when \emph{Pilot Event Start} is used, press the \emph
{Pilot Event} button, wait for \emph{PEV start wait time} and then cross the
start line within \emph{PEV start window} minutes.

\tip It may be useful to have \emph{Start open} and/or \emph{Start reach}
 infoboxes visible to effectively use the \emph{PEV Start} procedure.


\section{Thermals}

While climbing in thermals, a marker is automatically generated showing the
thermal location on screen.    The positions of the last 20 thermals are
stored until the end of the flight.
\sketch{figures/thermalhistory.png}
This thermal history is accessible through the map
element functions in the same way as markers or waypoints.


\section{Glide range line}\label{sec:reach}

A reachable glide `footprint' is displayed on the map display as a
black and white dashed line, indicating where the glider would descend
through the terrain clearance height.  The reach shows clearance
tracks extending in all directions, optionally including paths around
terrain obstructions.  This feature is useful in assessing range with
respect to topography when searching low for lift, and when flying in
mountainous areas.

Reach calculations may be configured \config{turningreach} to two levels of detail:
\begin{description}
\item[Straight line] If turning reach is disabled, then the reach shows the
 furthest distance the glider can fly in final glide in all directions without
 turning.  This reach appears as a closed ring around the glider.

\begin{center}
\includegraphics[angle=0,width=0.8\linewidth,keepaspectratio='true']{figures/reach1.png}
% CUTOUT SHOWING GLIDE RANGE FOOTPRINT.  NO TOPOGRAPHY, FULLSCREEN, NO TASK. TURNING=FALSE
\end{center}

\item[Turning] If turning reach is enabled, then the reach shows the
  greatest area the glider can reach in all directions, even allowing
  turns around obstructions.\footnote{The maximum number of turns is
    set to three, and no turns may be greater than 90 degrees.}  The
  reach area appears as a closed ring around the glider but may also
  include holes indicating mountain peaks that the glider cannot reach
  without further climb.

\begin{center}
\includegraphics[angle=0,width=0.8\linewidth,keepaspectratio='true']{figures/reach2.png}
% CUTOUT SHOWING GLIDE RANGE FOOTPRINT.  NO TOPOGRAPHY, FULLSCREEN, NO TASK. TURNING=TRUE
\end{center}

\end{description}

The display can be configured to additionally blur the unreachable area
outside the glide range. \config{gliderange}
The final glide path is checked for whether the glider clears terrain along
the path by the terrain clearance height (see Section~\ref{sec:safety-heights}).
If clearance is not attained, a red
cross appears on the map at the point where the violation occurs. If a target is
defined the calculation is done along the straight path to the target. If no 
target is defined the calculation is done along the current heading.

If reach is enabled, then the reachability of landable waypoints is used
by the abort task mode, alternate landable option lists and display of
landable waypoints on the map.

Note that task calculations are otherwise unaffected by reach
calculations --- for example, altitudes required as shown in the final
glide bar or task data as displayed in infoboxes do not take reach into account.

Furthermore, the reach calculations are used for footprint, landable
waypoint arrival heights, abort mode and the alternates dialogue.  The glider
performance and MacCready setting used in these calculations are configurable\config{reachpolar}:
\begin{description}
\item[Task] The MC value is that used in the task.
\item[Safety MC] A configurable, typically low MC value is set by the user to set
  performance near, but slightly degraded, to best glide performance. The amount of safety 
  in the reach calculation is then the gap between best glide performance and the glide
  performance following the safety MC speed to fly. 
\end{description}

\section{Status tabs `\emph{Flight}' and `\emph{Time}'}\label{sec:flight-status}

The status dialogue is a multi-tabular dialogue giving overview information on the 
flight, system, task, rules and times.
These are information, and hence cannot be modified by the user.
It is accessed via the gesture ``S'', or with the button menu.
\gesture{Left - Down - Right - Down - Left}
\begin{quote}
\bmenug{Info 2}\blink\bmenug{Status}
\end{quote}

\subsection*{Flight}
Select the tabular page `\emph{Flight}'. 
The flight status dialogue shows the status of the aircraft's locality.
It shows the location of the aircraft, the
maximum height gain, and nearest waypoint it's bearing and distance.
\sketch{figures/status-flight.png}

You may find this function useful when you need to report your
location to others.

\subsection*{Times}\label{sec:time-status}
Select the tabular page `\emph{Times}'. 
It shows the local time, UTC time and date, flight duration, takeoff and landing time, and
the local sunrise and sunset times.

Note that the values in the Status dialogue 
are static once the particular dialogue page is displayed. 
\sketch{figures/status-times.png}
That is, position, times, etc.\ do not update while the page is displayed.
To see the updated values, it is necessary to select a different tabular, 
then return to the previous dialogue to see the new values.


\section{Route}\label{sec:route}

XCSoar can plan paths around terrain and airspace obstacles in three
dimensions from the aircraft to the destination.  Such a path is known
as a route.  The height of the destination is the arrival height for
final waypoints, or may be higher for intermediate waypoints, as
dictated by the task system as required to complete the task.  Route
planning functions in normal ordered task mode, abort mode and goto
mode.

\begin{center}
\includegraphics[angle=0,width=0.8\linewidth,keepaspectratio='true']{figures/route3.png}
\end{center}

Routes take into account the glider polar performance and are
calculated to be optimal in the sense of minimum time.  By default,
route calculation is disabled, and can be enabled for terrain only or
terrain and airspace avoidance. \config{routemode}

Terrain is avoided vertically by the terrain safety height,\config{safetyterrain}
with no additional lateral clearance imposed.
Valid routes may result in the aircraft arriving at the destination
higher than the minimum height --- such as can occur when the
destination is just beyond a steep mountain.

Airspace is avoided horizontally by a buffer of approximately 250 m,
with no additional vertical clearance imposed.  Valid routes may fly
below or above airspace.

If MacCready is positive, then climbs are optionally allowed
 in the computed routes.  The top of the climb
may be limited to 500 m above the higher of the start and destination
ceiling, or increased to the ceiling defined by the thermal ceiling.
\config{routeceiling}  Climbs above the higher of the start and
destination altitude are penalised by a slower climb rate than the
actual MacCready value.

Some approximations and limitations of the route planning system are as follows:
\begin{itemize}
\item Where climbs are necessary (and permitted) to reach the destination,
the climbs are assumed to occur at the start of the route.
\item Climb-cruise segments are assumed to occur at constant altitude,
equivalent to many small climbs distributed along the path.
\item Individual turns between path segments greater than 90 degrees
  are not permitted.
\item Failures of the solver result in the route reverting to direct flight
from the aircraft location to the destination.
\end{itemize}
